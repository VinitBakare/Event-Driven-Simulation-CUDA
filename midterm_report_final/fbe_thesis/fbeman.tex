%%%%%%%%%%%%%%%%%%%%%%%%%%%%%%%%%%%%%%%%%%%%%%%%%%%%%%%%%%%%%%%%%%%%%%
%%
%% This LaTeX file typesets the User's Guide for fbe_tez.sty.
%%
%%%%%%%%%%%%%%%%%%%%%%%%%%%%%%%%%%%%%%%%%%%%%%%%%%%%%%%%%%%%%%%%%%%%%%
%%
%% COPYRIGHT 1999, 2001, 2003 by
%% Feza Kerestecioglu <kerestec@boun.edu.tr>
%%
%% Copying of part or all of any file in the fbe_tez package is
%% allowed under the following conditions only:
%% (1) You may freely distribute unchanged copies of the files. Please
%%     include the documentation when you do so.
%% (2) You may modify a renamed copy of any file, but only for your
%%     personal use.
%% (3) You may copy fragments from the files, for personal use only
%%     and as long as credit is given where credit is due.
%%
%% You are NOT ALLOWED to take money for the distribution or use of
%% these files or modified versions or fragments thereof.
%%
%%%%%%%%%%%%%%%%%%%%%%%%%%%%%%%%%%%%%%%%%%%%%%%%%%%%%%%%%%%%%%%%%%%%%%
%
\documentclass[12pt]{article}
%
% Pagestyle
%
\oddsidemargin9.6mm
\evensidemargin9.6mm
\topmargin-1cm
\headheight20pt
\textwidth155mm
\textheight232mm
\pagestyle{myheadings}
%
% Macros
%
\newcommand\fbe{{\tt fbe\_tez}}
\newcommand\report{{\tt report}}
\newcommand{\bq}{\begin{quotation}\noindent}
\newcommand{\eq}{\end{quotation}}
\renewcommand{\arg}[1]{$\langle\mbox{\it #1}\rangle$}
%
% Title declarations
%
\title{{\Huge \fbe{\tt.sty}} \\ A \LaTeX\ Style for Theses}
\author{Feza Kerestecio\u glu \\{\em Department of
Electrical-Electronics Engineering}\\{\em Bo\u gazi\c ci University}}
\date{July 25, 2003 \\ Version 4.2}
%
\begin{document}
\maketitle
\tableofcontents
\section{Introduction}
The style \fbe\ is for typesetting M.S. and Ph.D. theses in the
format required by the Institute for Graduate Studies in Science
and Engineering (or, in Turkish, {\em Fen Bilimleri
Enstit\"us\"u}, FBE), Bo\u gazi\c ci University. Although FBE has
not defined a compulsory format for the thesis proposals, you can
also use \fbe\ to typeset a Ph.D. proposal as well. The \fbe\
package is based on the format described in the manual
\cite{fbeman} by FBE for the preparation of theses, which is
valid as of July 2000. This version of \fbe\ (v4.2) is prepared so
as to comply with the requirements of the format which has been
updated in 2003 by FBE. Hence, it renders all the previous
versions obselete, since they were pertaining to a format which
is not valid anymore.

The \fbe\ style is based on the \report\ class of \LaTeX. Some
commands which apply to \report\ were changed so as to comply with
the requirements of FBE. There are also additional commands provided
by \fbe\ to typeset pages which you would not find in an ordinary
report but which must be included in the theses submitted to FBE
(e.g., a page of approval and an abstract in Turkish).

\section{How to Get and Invoke \fbe}
The style package \fbe{\tt .zip} can be obtained from the
personal web site of the author ({\tt
http://khas.edu.tr/\verb+~+kerestec/\fbe.zip}). After downloading,
when you unzip the file \fbe{\tt.zip}, five files will be
generated:
\begin{description}
\item[{\tt fbeman.tex}] The \LaTeX\ source of this user's guide.
\item[{\tt \fbe.sty}] The \LaTeX\ style file. This file should be
   placed in a directory where your \TeX-input files reside.
\item[{\tt sampthes.tex}] A sample \LaTeX\ file which uses the \fbe\
   package as a style. You can use this file as a template to type
   in your thesis.
\item[{\tt readme.txt}] A text file in which the corrections and
   changes in the newer versions of \fbe\ are reported.
\item[{\tt format.pdf}] A pdf file of the booklet describing the format
   for theses, namely \cite{fbeman}.
\end{description}

The \fbe\ style can be invoked directly as a package. As \fbe\ is
based on the report class, you have to start your \LaTeX\ source with
the command
\begin{verbatim}
     \documentclass[12pt]{report}
\end{verbatim}
which must be followed by
\begin{verbatim}
     \usepackage{fbe_tez}
\end{verbatim}
if you are using \LaTeXe.

The \fbe\ does not make use of new features introduced in the
\LaTeXe. Therefore, you can also use it with \LaTeX~2.09. The users
of \LaTeX~2.09 should include \fbe\ as an optional parameter in the
\verb/\documentstyle/ command as
\begin{verbatim}
     \documentstyle[12pt,fbe_tez]{report}
\end{verbatim}

\section{New Commands and Environments}
\subsection{Title Page}
The title pages for M.S. and Ph.D. theses can be obtained by two
different commands, namely
\begin{verbatim}
     \makemstitle
\end{verbatim}
or
\begin{verbatim}
     \makephdtitle
\end{verbatim}
Further, \fbe\ (version 4.0 or later) makes the
\begin{verbatim}
     \makeproposaltitle
\end{verbatim}
command available to you, which can be used to generate a title page
in your (Ph.D.) thesis proposal. The pages typeset by these commands
look very much the same, save that the name of the relevant degree,
program or other standard wording appears on either type of titlepage.
The obvious place for these commands is right after the
\verb/\begin{document}/ command, since the title is the leading
page of any thesis.

All three commands above assume that you have defined the title and
author of the document in the preamble by the corresponding \LaTeX\
commands, namely \verb/\title{/\arg{title}\verb/}/ and
\verb/\author{/\arg{name}\verb/}/.
You also have to declare the degree(s) of the author and the
submission year of the thesis in the preamble, since these information
should appear in the titlepage. This is done by the
\bq
\verb/\degree{/\arg{text}\verb/}/
\eq
and
\bq
\verb/\subyear{/\arg{year}\verb/}/
\eq
commands. If the author has more than one degree, which is typically
the case for a Ph.D. candidate, they must be separated in the argument
of the \verb/\degree/ command by \verb/\\/.

An additional information needed for generating the titlepage of an
M.S. thesis is the program for which M.S. degree the author is going
to submit the thesis. Similar to the \verb/\degree/ command, this
should be declared by a
\bq
\verb/\program{/\arg{text}\verb/}/
\eq
command.

\LaTeX\ will warn you if you do not declare any degree for the author.
Nevertheless, the titlepage will be typeset even if this information
is missing. The same holds for the program information in the case of
an M.S. thesis. On the other hand, the default value for the argument
of \verb/\subyear/ is the year of the current date. Therefore, you
need to declare it only if the submission date is going to be
different than the year when you are typesetting the thesis.

\subsection{Approval Page}
A page of approval can be generated simply by the
\begin{verbatim}
     \makeapprovalpage
\end{verbatim}
command. The information which is going to appear in the approval
page is also to be given in the preamble. The date of approval has to
be specified as the argument of a
\bq
\verb/\dateofapproval{/\arg{date}\verb/}/
\eq
command. \LaTeX\ will warn you you if you do not declare any date of
approval.

On the other hand, you should specify the members of the examining
committee by the commands
\bq
\verb/\supervisor{/\arg{title-and-name}\verb/}/\\
\verb/\cosuperi{/\arg{title-and-name}\verb/}/\\
\verb/\cosuperii{/\arg{title-and-name}\verb/}/\\
\verb/\examineri{/\arg{title-and-name}\verb/}/\\
\verb/\examinerii{/\arg{title-and-name}\verb/}/\\
\verb//\vdots\\
\verb/\examinerv{/\arg{title-and-name}\verb/}/
\eq

The \verb/\makeapprovalpage/ command will typeset the arguments of
each one of these commands as a separate examiner name. The
difference between the \verb/\supervisor/ and \verb/\examineri/, etc.
is that a phrase like `(Thesis Supervisor)' is typeset under the name
which is declared as the supervisor. Similarly, `(Thesis
Co-supervisor)' appears under the names given as the arguments of the
\verb/\cosuperi/ and \verb/\cosuperii/ commands. Regardless of the
order you put these commands in the preamble, the order of the typeset
examiner list is going to be as follows: First the supervisor, then
the co-supervisor(s) (if any) and then the examiners in the order
implied by the command names (that is, first the one specified as
\verb/\examineri/, then \verb/\examinerii/ and so on).

You can declare a supervisor and a (or, even two) co-supervisor(s) at
the same time if you wish. Nevertheless, I think, if your thesis has
been supervised by two people, it would be more appropriate to call
{\em both\/} of them as co-supervisors. You will get a \LaTeX\ warning
if neither a supervisor name nor any co-supervisor names are given.
But the approval page will still be generated even if this information
is missing.

The correct place for the \verb/\makeapprovalpage/ command is
certainly right after the command which generates the titlepage, as
required by FBE \cite{fbeman}.

\subsection{Acknowledgements and \"Ozet}
Two new environments are defined by the \fbe\ style for typesetting
the pages where the acknowledgements and ``\"ozet" (abstract in
Turkish) are going to appear. They work similar to the standard
{\tt abstract} environment. That is, you type
\bq
\verb/\begin{acknowledgements}/ \\
\verb//\vdots \\
\verb//$\langle\mbox{\it text of acknowledgements page}\rangle$\\
\verb//\vdots \\
\verb/\end{acknowledgements}/
\eq
or
\bq
\verb/\begin{ozet}/ \\
\verb//\vdots \\
\verb//$\langle\mbox{\it text of turkish abstract}\rangle$\\
\verb//\vdots \\
\verb/\end{ozet}/
\eq

The format specified by FBE \cite{fbeman} requires that the title
of the thesis appears in the Abstract and \"Ozet pages. Moreover,
and naturally, the title should appear in the \"Ozet Page as in
Turkish. Therefore, the title in Turkish has to be given in the
preamble as the argument of a
\bq
\verb/\turkcebaslik{/\arg{title in Turkish}\verb/}/
\eq
command.

According to the FBE Thesis format, you have to put your
acknowledgements after the approval page. The \fbe\ style also
provides you the {\tt preface} and {\tt foreword} environments, which
work exactly in the same way. Consult \cite{fbeman} for the correct
order of appearance of all these pages.

\subsection{List of Symbols}
To typeset a list of symbols section you type in your source file
\bq
\verb/\begin{symbols}/ \\
\verb/\sym{/\arg{symbol}\verb/}{/\arg{description}\verb/}/\\
\verb//\vdots \\
\verb/\sym{/\arg{symbol}\verb/}{/\arg{description}\verb/}/\\
\verb/\end{symbols}/
\eq
where the arguments of each \verb/\sym/ command is a symbol to be
listed and its description. This command will line up the symbols and
their descriptions into two columns. You may change the indentations
of these columns by redefining the lenghts \verb/\symtabi/ and
\verb/\symtabii/ using length-changing commands like \verb/\setlength/
or \verb/\addtolength/. Consult a \LaTeX\ manual (e.g. \cite{latex})
for the usage of these commands. The default values of \verb/\symtabi/
and \verb/\symtabii/ are {\tt 1.0em} and {\tt 10em}, respectively.

Moreover, to insert a blank line in the list, which you might need to
separate the Latin symbols from the Greek ones or the symbols list
from an abbreviations list, you can use a \verb/\sym/ command with
empty arguments, i.e., \verb/\sym{}{}/.

The \verb/symbols/ environment generates a list with the heading
as ``LIST OF SYMBOLS". You can use the environments
\verb/abbreviations/ and \verb/symabbreviations/, to generate
lists with headings ``LIST OF ABBREVIATIONS" or ``LIST OF
SYMBOLS/\-ABBREVIATIONS", respectively.

\subsection{Bibliographies}
In addition to the \LaTeX\ environment {\tt thebibliography}, which
has been reshaped by the \fbe\ style according to the instructions in
\cite{fbeman}, there are three new bibliography-making environments.

In \cite{fbeman}, two possible forms of bibliography referencing are
mentioned. One of them is by referring them with numbers in square
brackets (sometimes called the {\em IEEE\/\footnote{Institute of
Electrical and Electronics Engineers}-style} referencing). The other
one is by using the last name of the first author and the year of
publication in parenthesis. This is known as the {\em Harvard-style\/}
referencing. Also, according to \cite{fbeman}, a bibliographical list
of {\em not-cited references} can also be included in the thesis. With
the bibliography typesetting environments of the \fbe, it is possible
to generate both bibliography lists in either format.

The standard {\tt thebibliography} environment generates a
bibliography titled `REFERENCES' and which is almost in the
IEEE-style. Almost; because the reference numbers in the list are not
put into the square brackets, but followed by a period in the way
described in \cite{fbeman}. The standard cross-referencing method of
\LaTeX\ for bibliographical items still works. That means, each
reference item is specified after a
\bq
\verb/\bibitem{/\arg{key}\verb/}/
\eq
so that you can cite the reference using the {\em key\/} of the
reference by a \verb/\cite{/\arg{key}\verb/}/ command somewhere else
in the text.

To generate the reference list in Harvard style, you type
\bq
\verb/\begin{harvardbibliography}/ \\
\verb/\item/ $\langle\mbox{\it reference item}\rangle$\\
\verb//\vdots \\
\verb/\item/ $\langle\mbox{\it reference item}\rangle$\\
\verb/\end{harvardbibliography}/
\eq
Similarly, you can type
\bq
\verb/\begin{bibnotcited}{/\arg{widest-label}\verb/}/ \\
\verb/\item/ $\langle\mbox{\it reference item}\rangle$\\
\verb//\vdots \\
\verb/\item/ $\langle\mbox{\it reference item}\rangle$\\
\verb/\end{bibnotcited}/
\eq
to typeset a bibliography list titled `REFERENCES NOT CITED' in the
IEEE format (well, almost).  As in the {\tt thebibliography} command,
the argument of the {\tt bibnotcited} environment specifies the widest
number which a bibliographical item in the list can have. Finally,
typing
\bq
\verb/\begin{harvardbibnotcited}/ \\
\verb/\item/ $\langle\mbox{\it reference item}\rangle$\\
\verb//\vdots \\
\verb/\item/ $\langle\mbox{\it reference item}\rangle$\\
\verb/\end{harvardbibnotcited}/
\eq
will generate you a bibliography list titled `REFERENCES NOT CITED' in
Harvard style.

Note that the environments provided by \fbe\ use the \verb/\item/
command to declare each reference item, instead of \verb/\bibitem/.

\section{Modifications to Report Style}
Apart from the above commands which are introduced by the \fbe\ style,
several commands and environments of \report\ style have been
redefined to generate a typesetting format which obeys the
requirements of FBE. Although most of these changes are transparent to
the user, some of them are listed below just to inform the \TeX
nicians:
\begin{itemize}
\item The headings of the sectional units from chapter level down to
  paragraph level have been reformatted according to \cite{fbeman}.
  The headings for parts and subparagraphs have not been changed in
  any way. Because, a sectional unit to be called as a `Part' is not
  expected to appear in a thesis or submitted to FBE. On
  the other hand, nothing is specified about the format of paragraphs
  and subparagraphs (that is, fifth and sixth level headings) in
  \cite{fbeman}. Therefore, subparagraphs are typeset as in the
  ordinary \report\ style of \LaTeX. Nevertheless, you are not
  expected to use these items in a thesis anyway.
\item To the contrary of the classical \report\ style, there is a
  period after the numbers of the sectional units in the headings.
\item The first paragraph after a heading starts with a paragraph
  indentation.
\item All numbered sectional units (from {\tt chapter}s to {\tt
  subsubsection}s) are included in the Table of Contents (except,
  of course, the entry for the Table of Contents itself). First level
  headings, which are not numbered, such as List of Tables, etc. are
  included in the table of contents directly. You do not need the
  \verb/\addtocontents/ command to make them appear in the Table of
  Contents any more. Further, the indentations of the lines of the
  Table of Contents have been rearranged taking into account the
  periods which now appear after the numbers of the sectional units.
\item Figure and table captions are reformatted according to the FBE
  regulations \cite{fbeman}.
\item Proper pagestyle, margins, line spacing and interparagraph spacing
  are provided. The line spacing is 1.5, except in footnotes and
  quotations, where single line spacing has to be used. There is an extra
  1.5 space between the paragraphs. Also note that the
  \verb/\raggedbottom/ command is in effect. Therefore, the bottom
  margin might vary a bit from page to page. You can invoke the
  \verb/\flushbottom/ command in the preamble of your document if you
  wish. Nevertheless, this is not recommended for theses, which
  include pages with long formulae or large figures and are sparse in
  regular text, since the spacing in such pages might look ugly.
\item The spacings for displayed formulae are also taken care of by
  \fbe. Nevertheless, you can use the \TeX\ commands
  \verb/\abovedisplayskip/, \verb/\belowdisplayskip/,
  \verb/\abovedisplayshortskip/ and \verb/\belowdisplayhortskip/  for
  further 'fine tuning' if you wish. You have to use these commands
  {\em after} your \verb/\begin{document}/ command.
\item Footnotes are numbered consecutively throughout the whole
  thesis. If you would like to use the number of the recent footnote
  in any way, note that the footnotes are numbered by a new counter
  named {\tt thsfootcnt} (whose value can be printed by
  \verb/\thethsfootcnt/ command). This change is needed because the
  \report\ style resets the {\tt footnote} counter whenever a new
  chapter starts.
\item The {\tt appendix} command is redefined. As in the standard
  \report\ style it still changes the way sectional units are
  numbered. That is, the chapter numbers are put as A, B, etc. As a
  modifiation, now it also produces a line in the Table of Contents
  which starts with the word `APPENDIX' and the number of the chapter.
  So, after using the appendix command, you do not have to specify the
  optional parameter of the {\tt chapter} command to obtain the entry
  in the Table of Contents as required by FBE. Therefore, you will
  start the part of your source file where you write your appendices
  simply as
\bq
\verb/\appendix/ \\
\verb/\chapter{/\arg{appendix title}\verb/}/ \\
\verb//\vdots \\
\eq
\end{itemize}

\section{Bugs and Warnings}
You should be warned about the following bugs and drawbacks of \fbe.
\begin{enumerate}
\item Although a {\em co}-supervisor never comes alone, you will not
  get a warning if you specify no supervisor name and only one
  co-supervisor name.
\item The capitalization of the chapter headings are not provided in
  some cases. Therefore, you have to type the chapter names in the
  correct case as they should appear in the document.
\item If an appendix (say, the first one) does not have a title, the
  chapter heading will appear as `APPENDIX A:', i.e., with a column
  after the chapter number. Therefore, I strongly recommend you to
  title all appendix chapters. I think, this should not be considered
  as a bug, since all chapters (including the appendices) should have
  a title.
\item To invoke the correct pagenumbering, do not forget to use the
\begin{verbatim}
     \pagenumbering{roman}
\end{verbatim}
  command right after \verb/\begin{document}/ and
\begin{verbatim}
     \pagenumbering{arabic}
\end{verbatim}
  after the \verb/\chapter/ command which starts the first chapter of
  the thesis.
\item Note that \fbe\ does not support any document class or
  style other than \report\ and {\tt 12pt}. These are the only class and
  styles which you will need to typeset your thesis anyway.
\end{enumerate}

The author will very much appreciate it if you inform him on other
bugs, or suggestions you might have, by an e-mail to {\tt
kerestec@boun.edu.tr}.

\begin{thebibliography}{9}
\bibitem{fbeman} The Institute for Graduate Studies in Science and
  Engineering, {\em Format for Theses},
  Bo\u gazi\c ci University, Istanbul, 2002.
\bibitem{latex} Lamport, L., {\em A Document Preparation System:
  \LaTeX}, Addison-Wesley, Reading, 1986.
\end{thebibliography}
\end{document}
%
% End of fbeman.tex
